\section{Pasta Fresca}
\subsection{Ingredienti}
\begin{itemize}
\item 400 gr farina  
\item 4 uova  
\item 1 cucchiaino di olio  
\item 1 pizzico di sale
\end{itemize}
\subsection{Procedimento}
\begin{enumerate}
\item  Versare tutti gli ingredienti nell'impastatrice e lavorare con il gancio a bassa velocità per circa 10 minuti.  
\item  Dividere la pasta, schiacciarla un po' con la mano e tirarla con la macchinetta. Il primo passaggio è meglio ripeterlo tre volte girando ogni volta la pasta di 90 gradi. Non tirare la pasta troppo fine.
\end{enumerate}
\subsection{Note}
\begin{itemize}
\item Al posto di 100 gr di farina è possibile utilizzare una miscela di farina e semola di grano duro rimacinata nelle proporzioni 50/50, 30/70 o anche solo semola.   
\item Per dare alla pasta un gusto più rustico si pu ò spalmare la semola sulle strisce prima di passarle sotto il rullo: in questo modo la pasta diventa più ruvida.  
\item Per fare le pappardelle lasciare asciugare un po' la pasta quindi arrotolare le strisce con un raggio largo e tagliarle con un coltello molto affilato; srotolare subito e fare asciugare. Cuocere le pappardelle per 5 minuti. È possibile non farle asciugare: in questo caso vanno bollite per 1 minuto soltanto.
\end{itemize}
