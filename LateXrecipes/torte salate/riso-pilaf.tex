\section{Riso Pilaf}
\subsection{Ingredienti}
\begin{itemize}
\item Due tazze di riso  
\item Due tazze di brodo di verdura  
\item 1 cipolla piccola  
\item 1 bicchiere di vino bianco  
\item Olio  
\item Sale
\end{itemize}
\subsection{Procedimento}
\begin{enumerate}
\item  Accendere il fuoco a 200 C°.  
\item  Portare ad ebollizione il brodo.  
\item  In una pirofila soffriggere la cipolla con l'olio, aggiungere il riso e tostarlo. Sfumare con il vino già caldo.  
\item  Versare il brodo bollente, far riprendere il bollore e coprire con un foglio di carta argentata.  
\item  Cuocere in forno per 16'. Per un riso più tostato togliere la carta argentata gli ultimi 2 minuti.
\end{enumerate}
\subsection{Note}
\begin{itemize}
\item Per fare prima al posto del soffritto si pu ò usare una piccola cipolla tagliata in due oppure due scalogni, magari con infilati dei chiodi di garofano.  
\item Con l'aggiunta di tonno e piselli diventa un ottimo piatto unico. Tonno e piselli vanno aggiunti al riso dopo averlo soffritto; a fine cottura aggiungere olio crudo e prezzemolo tritato.  
\item Ottimo anche con sugo e polpettine.  
\item La ricetta originale prevede il doppio dell'acqua rispetto al riso ma in questo modo il riso diventa più tostato.
\end{itemize}
