\section{Riso Pilaf Agli Asparagi}
\subsection{Ingredienti}
\begin{itemize}
\item 1,5 kg asparagi verdi sottili  
\item 70 gr burro  
\item 50 gr farina  
\item 400 gr panna  
\item 1 bicchiere di latte  
\item 5 cucchiai di parmigiano  
\item Poco dado  
\item Sale
\end{itemize}
\subsection{Procedimento}
\begin{enumerate}
\item  Lessare gli asparagi in poca acqua salata fino a quando saranno appena morbidi. Scolarli tenendo da parte l'acqua di cottura. Mettere da parte 15-20 punte lunghe circa 4-5 cm e frullare il resto, eventualmente aggiungendo un po' di brodo di cottura.  
\item  Procedere alla preparazione del riso pilaf come al solito ma lasciandolo più morbido (una tazza e mezza di liquido ogni tazza di riso) e utilizzando l'acqua di cottura degli asparagi per il brodo.  
\item  Scaldare il latte con la panna. In un pentolino col fondo spesso sciogliere il burro e poi aggiungere tutta insieme la farina sciogliendola con una frusta. Aggiungere la panna e il latte sempre mescolando fino ad ottenere una besciamella liscia e compatta.   
\item  Aggiungere un po' di dado, il parmigiano, il sale e il frullato di asparagi e mescolare.  
\item  Quando il riso è pronto scaldare la salsa e usarne metà da mescolare al riso e metà per coprirlo.   
\item  Decorare con le punte di asparagi.
\end{enumerate}
