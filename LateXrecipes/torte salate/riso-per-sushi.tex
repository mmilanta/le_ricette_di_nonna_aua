\section{Riso Per Sushi}
\subsection{Ingredienti}
\begin{itemize}
\item 100 gr riso per sushi (o vialone nano)  
\item 120 gr acqua  
\item 1 cucchiaio abbondante aceto di mele (o di riso)  
\item 1 cucchiaio abbondante zucchero  
\item 1 cucchiano sale
\end{itemize}
\subsection{Procedimento}
\begin{enumerate}
\item  Lavare il riso sotto l'acqua corrente lasciandolo anche a bagno fino a quando lascia l'acqua limpida (tra 15 e 30').   
\item  Lasciarlo quindi riposare su uno scolino per circa 15'.   
\item  Nel frattempo preparare la riduzione mettendo sul fuoco un pentolino con aceto, zucchero e sale. Scaldare e mescolare con un cucchiaio di legno per sciogliere lo zucchero e il sale ma non fare bollire per non fare evaporare l'acidità dell'aceto.  
\item  Versare il riso nella pentola con l'acqua e portarlo ad ebollizione a fuoco vivo. Al bollore abbassare e cuocere a fuoco bassissimo senza togliere il coperchio per 10-15'.  
\item  Togliere dal fuoco e lasciare riposare per altri 15' coperto con un canovaccio e col coperchio.  
\item  Versare in una ciotola larga (non di metallo) raffreddandolo con un coperchio usato come ventaglio e condire con la riduzione mescolando delicatamente con un mestolo di legno per non rompere i chicchi di riso.  
\item  Coprire con un canovaccio e consumare in giornata. Non va messo in frigorifero. 
\end{enumerate}
