\section{Risotto}
\subsection{Ingredienti}
\begin{itemize}
\item Due tazze di riso  
\item Quattro tazze di brodo  
\item 1 cipolla piccola  
\item 1 bicchiere di vino bianco  
\item Olio  
\item Sale
\end{itemize}
\subsection{Procedimento}
\begin{enumerate}
\item  Soffriggere olio e cipolla. Tostare il riso e sfumare col vino bianco caldo.   
\item  Aggiungere eventuali sapori e rosolare a fuoco vivace per qualche minuto fino a quando viene assorbito il condimento.   
\item  Abbassare il fuoco e aggiungere un mestolo di brodo bollente. Aspettare che il brodo venga assorbito dal riso prima di aggiungerne altro. Continuare così fino a cottura ultimata.  
\item  Il risotto viene ottimo con la pentola a pressione. Aggiungere il brodo bollente tutto insieme, chiudere la pentola e cuocere a fuoco vivace fino al fischio, poi abbassare e continuare la cottura per 4-5 minuti.  
\item  A fine cottura aggiungere altri eventuali sapori, mantecare con burro o formaggio e mescolare vigorosamente.  
\item  Lasciare riposare 5 minuti e servire.
\end{enumerate}
\subsection{Note}
\begin{itemize}
\item Ottimi trucchi per mantecare sono mescolare molto vigorosamente, aggiungere burro freddo di freezer, o frullare qualche cucchiaio di risotto e poi aggiungerlo al resto.  
\item Risotto ai funghi  
\item Risotto al barolo e castelmagno  
\item Risotto allo zafferano  
\item Risotto Martini dry, scampi e agrumi  
\item Risotto salsiccia, verza e zafferano  
\item Risotto patate e prezzemolo
\end{itemize}
