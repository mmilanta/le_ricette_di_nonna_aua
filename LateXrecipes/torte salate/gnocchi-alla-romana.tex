\section{Gnocchi Alla Romana}
\subsection{Ingredienti}
\begin{itemize}
\item 1 litro latte  
\item 250 gr semolino  
\item 30 gr burro  
\item 2 cucchiai parmigiano  
\item 2 uova  
\item Sale
\end{itemize}
\subsection{Procedimento}
\begin{enumerate}
\item  In una pentola portare ad ebollizione il latte con il sale.  
\item  Aggiungere a pioggia il semolino e cuocere a fuoco basso mescolando con un cucchiaio di legno fino a raggiungere il bollore. Continuare per 2-3 minuti dopo l'ebollizione e togliere dal fuoco.  
\item  Incorporare il burro, il parmigiano e le uova.  
\item  Bagnare una superficie, rovesciare l'impasto e stenderlo alto 1 cm. Lasciare raffreddare e ritagliare dei dischi.  
\item  Ungere una pirofila di burro e disporre i ritagli in un solo strato, quindi coprire con i dischi leggermente sovrapposti. Cospargere con dei fiocchetti di burro e del parmigiano.  
\item  Infornare per 1 ora a 200 C°.
\end{enumerate}
\subsection{Note}
\begin{itemize}
\item Prima della cottura gli gnocchi possono essere congelati.  
\item Per fare prima si pu ò versare il composto di uova, latte e semolino in uno stampo da ciambella molto unto e coprire con fiocchi di burro. Cuocere a 200 C° per 1 ora. Capovolgere e riempire il buco con piselli, funghi ecc.
\end{itemize}
