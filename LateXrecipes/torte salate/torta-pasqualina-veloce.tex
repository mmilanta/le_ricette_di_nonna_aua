\section{Torta Pasqualina Veloce}
\subsection{Ingredienti}
\begin{itemize}
\item 1 confezione di pasta brisé o pasta sfoglia  
\item 500 gr spinaci  
\item 300 gr ricotta  
\item 3 uova  
\item 200 gr di latte  
\item Parmigiano  
\item Sale 
\end{itemize}
\subsection{Procedimento}
\begin{enumerate}
\item  Foderare una tortiera con la pasta pronta e bucherellarla con la forchetta.  
\item  Bollire gli spinaci, scolarli, strizzarli bene schiacciandoli con il dorso di un cucchiaio e sminuzzarli col coltello.  
\item  Preparare un composto con la ricotta, la verdura, le uova, il latte, il parmigiano e il sale. Versare il composto nella tortiera sopra la pasta.   
\item  Infornare a 200 C° per circa 40 minuti. Coprire eventualmente con un foglio di alluminio per evitare che bruci sopra ed assicurarsi che cuocia bene sotto.  
\item  Servire direttamente dalla pirofila. 
\end{enumerate}
\subsection{Note}
\begin{itemize}
\item Per un risultato più simile all'originale coprire con un'altra sfoglia di pasta.  
\item La ricetta originale prevede l'utilizzo di pasta pazza e la copertura con molti strati soffiati con una cannuccia per farli gonfiare. Inoltre andrebbe usata la prescinseua al posto della ricotta.  
\item Una ricetta alternativa consiste nel versare sopra la pasta separatamente prima la verdura e poi il composto di ricotta.
\end{itemize}
