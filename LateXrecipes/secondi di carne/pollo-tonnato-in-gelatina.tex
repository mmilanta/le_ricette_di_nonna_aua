\section{Pollo Tonnato In Gelatina}
\subsection{Ingredienti}
\begin{itemize}
\item 1 pollo intero   
\item 250 gr. tonno sott'olio  
\item 150 gr. prosciutto cotto  
\item 4 cucchiai maionese  
\item 3 acciughe salate   
\item 1 cucchiaio capperi sotto sale  
\item 1 limone  
\item Olive nere  
\item Sottaceti  
\item 1 limone  
\item 1 dado   
\item Gelatina (dose per 500 gr.)
\end{itemize}
\subsection{Procedimento}
\begin{enumerate}
\item  Pulire il pollo. Lavarlo bene dentro e fuori e metterlo in una pentola coperto a metà di acqua. Far bollire col coperchio per circa 45 minuti o fino a quando il pollo diventa tenero. Si pu ò usare anche la pentola a pressione aggiungendo solo 3 dita d'acqua e calcolando circa 15 minuti dal fischio.   
\item  Lasciare raffreddare il pollo in uno scolapasta quindi disossarlo cercando di lasciare dei pezzi di polpa intera.   
\item  Tritare i pezzetti piccoli di pollo insieme al tonno, alle acciughe, ai capperi, al prosciutto e al succo di un limone: si deve ottenere una crema omogenea a cui va aggiunta la maionese mescolando piano a mano.  
\item  Su un piatto di portata un po' profondo o su una pirofila spalmare tutta la crema di pollo; disporre sopra in bell'ordine i pezzi di pollo; aggiungere le fettine di limone tagliate sottili senza buccia, le olive nere e i sottaceti.  
\item  Preparare la gelatina con l'aggiunta del succo di 1 limone seguendo le istruzioni della confezione, lasciarla raffreddare - non troppo altrimenti indurisce, circa 10 minuti in frigorifero - e versarla sopra il piatto raffreddato anch'esso in frigorifero. Eventualmente procedere in due stadi.  
\item  Lasciare il piatto finito in frigorifero fino a quando tutta le gelatina non sarà solidificata (circa mezz'ora).
\end{enumerate}
