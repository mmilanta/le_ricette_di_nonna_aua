\section{Brasato Clementina}
\subsection{Ingredienti}
\begin{itemize}
\item 1,5 kg carne da brasato  
\item 1 litro di vino rosso corposo (nebbiolo o barolo)  
\item 1 cipolla  
\item 2 carote  
\item Sedano  
\item 2 foglie di alloro  
\item 2 spicchi aglio  
\item Salvia  
\item Rosmarino  
\item 4 bacche ginepro
\item 4 chiodi di garofano
\item 4 grani pepe nero  TODO check 4
\item Pinoli, mandorle e uvetta  
\item Olio
\end{itemize}
\subsection{Procedimento}
\begin{enumerate}
\item  In un contenitore di plastica o di vetro (non di metallo) marinare in frigo la carne con il vino e tutti i sapori per una notte.  
\item  Preparare un battuto con mandorle e pinoli e a piacere aggiungere l'uvetta ammorbidita.  
\item  Riportare la carne a temperatura ambiente ed asciugarla. Rosolarla su tutti i lati con un po' d'olio. Aggiungere la verdura, il battuto di mandorle e pinoli e poi piano piano il vino della marinatura. Coprire col coperchio e cuocere a fuoco basso per circa 5 ore girandola ogni tanto.  
\item  Quando la carne è molto tenera toglierla dalla pentola e lasciarla raffreddare in un piatto. Togliere l'alloro, il ginepro e il bastoncino del rosmarino dalla pentola e frullare il resto, quindi fare asciugare il sugo.   
\item  Tagliare la carne a fette e condirla col suo sugo.
\end{enumerate}
\subsection{Note}
\begin{itemize}
\item Viene bene anche senza marinatura e con meno sapori, l'essenziale è la cottura molto lunga.  
\item È più buono scaldato il giorno dopo.
\end{itemize}
