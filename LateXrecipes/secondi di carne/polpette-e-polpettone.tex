\section{Polpette E Polpettone}
\subsection{Ingredienti}
\begin{itemize}
\item 600 gr. carne di:   
\item 2 uova  
\item 1 cucchiaio colmo parmigiano  
\item Pane  
\item Latte  
\item Sale  
\item Prezzemolo  
\item Rosmarino  
\item Aglio
\end{itemize}
\subsection{Procedimento}
\begin{enumerate}
\item  Tritare la carne con eventuali sapori.  
\item  Tagliare il pane a pezzi, bagnarlo con il latte, lasciarlo ammorbidire per qualche minuto quindi scolarlo e strizzarlo bene.  
\item  Mescolare il trito con le uova, il pane, il parmigiano e il sale.   
\item  Formare le polpette o il polpettone.  
\item  Polpette  
\item  Formare le polpette con le mani pressarle bene.  
\item  Preparare un sugo di pomodoro e cuocerle lentamente e a lungo mescolando ogni tanto con delicatezza per non romperle.   
\item  In alternativa versare in una padella larga un po' d'olio. Quando è ben caldo aggiungere le polpette in un unico strato e lasciare soffriggere a fuoco abbastanza alto per qualche minuto fino a quando le polpette non fanno la crosticina. Girare con la spatola e terminare la cottura.  
\item  Aggiungere il pomodoro e il sale. Abbassare il gas e lasciare cuocere per circa altri 10 minuti fino a quando il sugo si è addensato.  
\item  Riscaldate sono ancora più buone.  
\item  Polpettone:  
\item  Formare 2 polpettoni.  
\item  Versare in una padella larga un po' d'olio. Quando è ben caldo aggiungere i polpettoni e lasciare soffriggere a fuoco abbastanza vivace per 4-5 minuti fino a quando i polpettoni non faranno la crosticina. Girare con la spatola e lasciare cuocere per altri 4/5 minuti.  
\item  Coprire col pomodoro e completare la cottura.
\end{enumerate}
\subsection{Note}
\begin{itemize}
\item Quando il polpettone è crudo ma già in forma si pu ò tagliarlo in due come un panino e metterci dentro una sorpresa: frittata, formaggio, affettato, ecc.  
\item Il polpettone pu ò essere preparato in bianco senza pomodoro.
\end{itemize}
