\section{Torta Delle Rose}
\subsection{Ingredienti}
\begin{itemize}
\item 500 gr farina  
\item 25 gr lievito di birra  
\item 2 uova  
\item 120 gr burro  
\item 150 gr zucchero  
\item 1 bicchiere abbondante di latte  
\item 2 bustine di vanillina  
\item 1 cucchiaino di malto o zucchero  
\item Zucchero a velo  
\item 1 pizzico sale
\end{itemize}
\subsection{Procedimento}
\begin{enumerate}
\item  Preparare un preimpasto col malto, il lievito, un po' di latte e 100 gr di farina presi dal totale. Lasciare riposare circa 30'.  
\item  Impastare il preimpasto con la farina rimanente, un pizzico di sale, metà zucchero e il latte rimanente. Aggiungere piano piano le uova e metà burro ammorbidito. Impastare fino a quando la pasta presenta una consistenza morbida ed elastica. Lasciare lievitare fino al raddoppio (da 1 a 3 ore).   
\item  Foderare di carta da forno uno stampo di 28-30 cm di diametro.  
\item  Preparare in una ciotolina il rimanente burro e lo zucchero con la vanillina. Mescolare con un cucchiaino fino ad ottenere una crema omogenea.  
\item  Dopo la prima lievitazione, dividere la pasta in 7 pagnotte e formare dei salamini da tirare con il mattarello fino ad ottenere dei rettangoli di circa 5x15 cm. Spalmare le strisce con la crema, arrotolarle a cilindro su sé stesse, chiuderle ad una estremità e metterle nello stampo in piedi, con l'estremità sigillata in basso, una accanto all'altra, come dei boccioli di rosa.  
\item  Fare lievitare una seconda volta fino a quando i boccioli riempiranno la teglia (circa 1 ora).  
\item  Cuocere a 180 C° per circa 20-25 minuti. Se la pasta tende a bruciare coprirla con un foglio di alluminio.
\end{enumerate}
