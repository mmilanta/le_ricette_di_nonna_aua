\section{Torta Di Taglierini Alla Modenese}
\subsection{Ingredienti}
\begin{itemize}
\item Per la pasta frolla  
\item 200 gr farina  
\item 100 gr zucchero  
\item 100 gr burro   
\item 1 tuorlo  
\item Per il ripieno  
\item 30 gr burro  
\item 100 gr zucchero  
\item 150 gr farina di mandorle  
\item 100 gr cedro e arancio candito  
\item 1 uovo   
\item Per i taglierini  
\item 100 gr farina  
\item 1 uovo
\end{itemize}
\subsection{Procedimento}
\begin{enumerate}
\item  Preparare la pasta frolla e lasciarla riposare mezz'ora in frigo avvolta nella pellicola. Preparare un uovo di taglierini.  
\item  Preparare il ripieno con la farina di mandorle, lo zucchero, il burro, metà canditi e un uovo. Deve venire un impasto morbido, se necessario aggiungere poca acqua.  
\item  Imburrare e infarinare uno stampo di 26 cm di diametro, foderarlo con la pasta frolla e versarvi il ripieno. Coprire con i taglierini e poi con un foglio di alluminio.   
\item  Infornare a 200 C° per 40'; a metà cottura togliere l'alluminio.  
\item  Spolverizzare con zucchero a velo e il resto dei canditi.
\end{enumerate}
\subsection{Note}
\begin{itemize}
\item Per fare prima preparare una doppia dose di pasta frolla e utilizzare la pasta per la copertura al posto dei taglierini. Alternativamente si possono usare taglierini pronti.  
\item Questa torta si conserva molto a lungo.
\end{itemize}
