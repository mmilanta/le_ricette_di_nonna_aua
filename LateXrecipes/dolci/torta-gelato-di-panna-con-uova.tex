\section{Torta Gelato Di Panna Con Uova}
\subsection{Ingredienti}
\begin{itemize}
\item 250 m panna da montare  
\item 3 uova  
\item 4 cucchiai zucchero  
\item 1 cucchiaio di marsala  
\item Pavesini / savoiardi / meringhe / zabaione
\end{itemize}
\subsection{Procedimento}
\begin{enumerate}
\item  Montare prima gli albumi a neve con una goccia di limone, prima da soli poi con metà zucchero. Quindi montare la panna e infine i tuorli con il restante zucchero.  
\item  Incorporare delicatamente dall'alto in basso la panna al composto d'uovo. Aggiungere da ultimo gli albumi a neve.  
\item  Versare in uno stampo imburrato uno strato di crema, poi uno strato di meringhe o biscotti leggermente inzuppati o zabaione e poi coprire con un altro strato di crema.   
\item  Congelare.  
\item  Prima di servire rovesciare la torta e decorare.  
\item  Volendo si pu ò foderare lo stampo con la pellicola: in questo modo è molto facile rovesciare la torta ma la parte sopra risulta irregolare per cui va coperta con la decorazione; per esempio ciuffi di panna.
\end{enumerate}
\subsection{Note}
\begin{itemize}
\item Prima di servirla lasciare la torta a temperatura ambiente per circa 30 minuti.  
\item Se si usano le meringhe queste vanno o sbriciolate o inserite in mezzo alla torta altrimenti non si riesce a tagliare bene le fette.  
\item Una volta congelata la torta pu ò essere rovesciata e decorata e quindi inserita nuovamente nel freezer.  
\item La ricetta originale prevede di montare gli albumi senza zucchero e montare tutto lo zucchero con i tuorli.
\end{itemize}
