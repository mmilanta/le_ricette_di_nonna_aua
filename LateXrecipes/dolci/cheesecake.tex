\section{Cheesecake}
\subsection{Ingredienti}
\begin{itemize}
\item 190 gr biscotti tipo digestive  
\item 80 gr burro  
\item 500 gr ricotta.  
\item 250 gr panna  
\item 110 gr zucchero  
\item 1 bustina di vanillina  
\item 2 cucchiai latte  
\item 12 gr gelatina in fogli  
\item Mettere a bagno la gelatina in acqua fredda per circa 10'.
\end{itemize}
\subsection{Procedimento}
\begin{enumerate}
\item  Foderare il fondo di uno stampo apribile (diametro 24 cm) con la carta da forno imburrata e bloccata con i bordi dello stampo. Imburrare bene anche i bordi dello stampo.  
\item  Tritare i biscotti e aggiungere il burro fuso. Amalgamare bene e distribuire l'impasto sul fondo dello stampo. Livellare con le mani e lasciare raffreddare in frigo.  
\item  Sbattere per 5' ricotta, zucchero e vanillina e montare la panna ben ferma. Scaldare il latte e sciogliervi la gelatina ben strizzata. Aggiungere il latte alla ricotta e mescolare per altri 2'. Aggiungere la panna e mescolare delicatamente con la frusta.   
\item  Versare il composto sul biscotto e lasciare in frigo per almeno 4 ore, meglio tutta la notte.
\end{enumerate}
\subsection{Note}
\begin{itemize}
\item Si pu ò sostituire la ricotta con lo yogurt magro. In questo caso alzare la dose di gelatina (14 gr).  
\item Tagliare la torta con un coltello con la lama scaldata.  
\item Per cheesecake con dentro composta di frutta aumentare leggermente la gelatina (15-18 gr).  
\item Per la cheesecake ai mirtilli scaldare 250 gr. di mirtilli con 100 gr. di acqua e 60 gr. di zucchero, aggiungere un foglio di gelatina e lasciare intiepidire. Versare sopra la torta già fredda e fare raffreddare. 
\end{itemize}
