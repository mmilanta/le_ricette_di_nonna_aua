\section{Ciaccia Toscana}
\subsection{Ingredienti}
\begin{itemize}
\item 500 gr farina 0  
\item 300 ml acqua  
\item 12 gr lievito di birra  
\item 1 cucchiaino di malto  
\item 100 gr zucchero  
\item 800 gr di uva fragola  
\item 8 cucchiai di olio  
\item 1 pizzico sale
\end{itemize}
\subsection{Procedimento}
\begin{enumerate}
\item  Sciogliere il lievito col malto e poca acqua tiepida e lasciare riposare per qualche minuto. Aggiungere la farina ed impastare col gancio. Lasciare lievitare fino al raddoppio.  
\item  Aggiungere 50 gr zucchero, 4 cucchiai di olio e un pizzico di sale e rimpastare.  
\item  Dividere la pasta in 2 parti, una leggermente più grande ed una più piccola, e tirare 2 sfoglie (circa 40 x 30). Per riuscire a spostare la sfoglia sulla teglia senza romperla conviene stenderla su carta da forno.  
\item  Oliare una teglia e adagiarvi la sfoglia più spessa. Coprire con 2/3 dell'uva, un cucchiaio di zucchero e un filo d'olio. Coprire con la seconda sfoglia, sigillando bene i bordi. Coprire con l'uva restante, spolverizzare con lo zucchero e cospargere con l'olio rimanente  
\item  Cuocere a 180° per circa 50-60 minuti. Gli ultimi 10 minuti cuocere con la resistenza in basso in modo da asciugare bene il fondo della ciaccia. 
\end{enumerate}
\subsection{Note}
\begin{itemize}
\item Per conservarla in frigo mettere l'uva solo fra le due sfoglie. In questo modo si possono sovrapporre più pezzi.  
\item Va bene anche la farina 00.  
\item Non usare teglie di ferro perché ossidano.
\end{itemize}
