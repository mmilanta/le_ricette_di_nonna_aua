\section{Galani}
\subsection{Ingredienti}
\begin{itemize}
\item 500 gr farina forte   
\item 3 uova  
\item 1 tazzina da caffè  
\item Vino bianco   
\item 1 cucchiaio di grappa  
\item 2 pizzichi di sale  
\item 50 gr di burro  
\item Raschiatura di limone   
\item Zucchero a velo abbondante (circa 2 buste)  
\item Olio di semi di arachide per la frittura
\end{itemize}
\subsection{Procedimento}
\begin{enumerate}
\item  Mescolare la farina, lo zucchero, il burro ammorbidito, le uova e il sale e impastare col gancio. Aggiungere la scorza di limone grattugiata finemente (altrimenti si incastra nei rulli della sfogliatrice). Impastare velocemente. L'impasto deve rimanere abbastanza duro (più o meno come la pasta fresca).  
\item  Avvolgere con la pellicola e lasciare riposare per circa mezz'ora a temperatura ambiente.  
\item  Tirare delle strisce sottili e con la rondella ondulata tagliare dei rettangoli abbastanza grandi e fare due tagli nel mezzo.  
\item  Friggere a 175-180 C° pochi pezzi per volta. Appena fritti spolverare con lo zucchero a velo.  
\item  Note: 
\end{enumerate}
\subsection{Note}
\begin{itemize}
\item Se non cuociono bene rimangono bianchi e perdono croccantezza.  
\item Non esagerare con lo zucchero a velo se no perdono croccantezza.  
\item Il riposo dell'impasto serve per facilitare le tirate ma una volta stesa la pasta va fritta il prima possibile per sviluppare le bolle. Conviene anche tagliare i galani man mano che si friggono.  
\item I liquidi (vino e grappa) dovrebbero essere in tutto circa 50 gr.  
\item Alcune ricette aggiungono all'impasto anche 50 gr di zucchero.  
\item Nonna Aua suggerisce di sostituire al burro 1 cucchiaio di olio.  
\item In alternativa alla farina forte utilizzare una miscela con 400 gr di Manitoba e 100 gr di farina 00.
\end{itemize}
