\section{Pomodori Gratinati}
\subsection{Ingredienti}
\begin{itemize}
\item 4 pomodori a grappolo sodi e maturi  
\item 55 gr pane grattato   
\item 20 gr acciughe  
\item Prezzemolo  
\item Capperi  
\item Basilico  
\item 1 spicchio di aglio  
\item Olio  
\item Sale
\end{itemize}
\subsection{Procedimento}
\begin{enumerate}
\item  Lavare ed asciugare i pomodori. Tagliarli a metà, estrarre la polpa interna e spremerli leggermente con le mani in modo da eliminare l'eccesso di liquido. Aggiungere un pizzico di sale e lasciarli capovolti su una gratella per circa 15' per eliminare l'acqua di vegetazione.  
\item  Mescolare insieme il pangrattato, le alici scolate e tritate, i capperi, l'aglio sminuzzato, il prezzemolo e il basilico tritati, l'olio, il sale ed eventualmente un po' del liquido di vegetazione dei pomodori versandolo poco alla volta sino a quando il ripieno non risulterà morbido.  
\item  Rivoltare i pomodori e metterli in una pirofila unta d'olio uno vicino all'altro. Coprirli con il trito e un filo d'olio.  
\item  Infornare a circa 180° per circa 30'. I pomodori dovranno diventare leggermente raggrinziti e la superficie ben dorata.
\end{enumerate}
