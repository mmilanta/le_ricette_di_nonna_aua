\section{Crema Catalana}
\subsection{Ingredienti}
\begin{itemize}
\item 1 lt latte  
\item 150 gr zucchero + qualche cucchiaio per la crosticina  
\item 6 tuorli  
\item 30 gr maizena  
\item 2 limoni 
\end{itemize}
\subsection{Procedimento}
\begin{enumerate}
\item  In una ciotola sciogliere la maizena con un bicchiere di latte freddo  
\item  Scaldare il rimanente latte, la buccia grattugiata dei due limoni e metà zucchero fino a sfiorare il bollore.  
\item  In una ciotola sbattere i tuorli con lo zucchero rimanente. Aggiungere il latte con la maizena, mescolare e aggiungere al latte caldo a fuoco spento.  
\item  Far riprendere il bollore e lasciare sobbollire per circa 2 minuti.  
\item  Versare negli stampini, lasciare raffreddare e mettere in frigo per almeno 4 ore, meglio tutta la notte.  
\item  Cospargere con un cucchiaio di zucchero e caramellizzare con l'apposito attrezzo.
\end{enumerate}
\subsection{Note}
\begin{itemize}
\item La ricetta originale prevede di utilizzare lo zucchero di canna per il caramello. Tuttavia lo zucchero di canna tende a bruciare facilmente mentre con lo zucchero bianco si ottiene un bell'effetto vetrificato.
\end{itemize}
