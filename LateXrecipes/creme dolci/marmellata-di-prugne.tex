\section{Marmellata Di Prugne}
\subsection{Ingredienti}
\begin{itemize}
\item 1 kg prugne sbucciate  
\item 250 gr. zucchero  
\item 1 limone
\end{itemize}
\subsection{Procedimento}
\begin{enumerate}
\item  Sterilizzare vasetti e coperchi in forno a 100 C° per circa un'ora.  
\item  Sobbollire a lungo la frutta con lo zucchero fino quasi ad ottenere la consistenza desiderata.  
\item  Aggiungere il limone e passare il minipimer.  
\item  Fare addensare sul fuoco. In questa fase la marmellata si attacca molto facilmente quindi va controllata spesso.  
\item  Versare la marmellata ancora bollente nei vasetti con l'apposito imbuto stando attenti a non sporcare i bordi del vasetto chiudere bene e fare raffreddare a testa in giù.
\end{enumerate}
\subsection{Note}
\begin{itemize}
\item Per un risultato più denso aggiungere un po' di pectina o una piccola mela.   
\item Si possono utilizzare anche albicocche.  
\item Con questa ricetta si ottiene una marmellata particolarmente poco dolce. In media le ricette suggeriscono di utilizzare da 500 a 750 gr di zucchero per kg di frutta.
\end{itemize}
