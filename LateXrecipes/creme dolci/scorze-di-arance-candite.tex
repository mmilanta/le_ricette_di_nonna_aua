\section{Scorze Di Arance Candite}
\subsection{Ingredienti}
\begin{itemize}
\item Arance biologiche con la buccia spessa  
\item Zucchero  
\item Acqua
\end{itemize}
\subsection{Procedimento}
\begin{enumerate}
\item  Lavare bene le arance e tagliare la buccia a spicchi e quindi a striscioline larghe circa 1 cm conservando anche la parte bianca.  
\item  Mettere le scorze in un pentolino di aqua fredda, portare al bollore e lasciare bollire per 5'. Scolarle e pesarle.  
\item  Mettere in una pentola col fondo spesso le scorze e lo stesso peso di zucchero e di acqua. Cuocere a fuoco basso fino a quando lo sciroppo formato dallo zucchero e dall'acqua non sarà completamente assorbito dalle scorze.  
\item  Con una pinza prendere una scorza alla volta e disporle su una grata, separate una dall'altra. Lasciarle per almeno 24 ore ad asciugare.  
\item  Passarle nello zucchero semolato e conservarle in una scatola di latta.
\end{enumerate}
\subsection{Note}
\begin{itemize}
\item Se le arance sono molto aromatiche o se non si gradisce il retrogusto di amaro, ripetere la fase di bollitura delle scorze altre 1 o 2 volte.  
\item Per un risultato speciale fondere del cioccolato fondente e inzupparci metà scorza. Fare asciugare su un foglio di carta da forno. 
\end{itemize}
